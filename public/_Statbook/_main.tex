\documentclass[a4paper,12pt,oneside]{book}
\usepackage[italian]{babel}
\usepackage[utf8]{inputenc}
\usepackage{textcomp}
\usepackage[parfill]{parskip} %Se necessatrio non indenta, ma inserisce spazio
\usepackage{graphicx}
\usepackage{hyperref}
\usepackage{amsmath} %To number equations

\usepackage{titling}
\newcommand{\subtitle}[1]{%
 \posttitle{%
 \par\end{center}
 \begin{center}\large#1\end{center}
 \vskip6.5em}%
}

\author{Andrea Onofri e Dario Sacco}
\date{Update: v. 1.0 (15/03/2021), compil. 2021-05-05}
\title{Metodologia sperimentale per le scienze agrarie}
\subtitle{}


%***************************************************************

%Specific RMarkdown
\usepackage{color}
\usepackage{fancyvrb}
\usepackage{longtable}
\usepackage{booktabs}
\providecommand{\tightlist}{%
  \setlength{\itemsep}{0pt}\setlength{\parskip}{0pt}}
\newcommand{\VerbBar}{|}
\newcommand{\VERB}{\Verb[commandchars=\\\{\}]}
\DefineVerbatimEnvironment{Highlighting}{Verbatim}{commandchars=\\\{\},fontsize=\small}
\usepackage{framed}
%\newenvironment{Shaded}{}{}
\newenvironment{Shaded}{\begin{snugshade}}{\end{snugshade}}
\definecolor{shadecolor}{RGB}{250,248,248}
\newcommand{\KeywordTok}[1]{#1}
\newcommand{\DataTypeTok}[1]{#1}
\newcommand{\DecValTok}[1]{#1}
\newcommand{\BaseNTok}[1]{#1}
\newcommand{\FloatTok}[1]{#1}
\newcommand{\ConstantTok}[1]{#1}
\newcommand{\CharTok}[1]{#1}
\newcommand{\SpecialCharTok}[1]{#1}
\newcommand{\StringTok}[1]{#1}
\newcommand{\VerbatimStringTok}[1]{#1}
\newcommand{\SpecialStringTok}[1]{#1}
\newcommand{\ImportTok}[1]{#1}
\newcommand{\CommentTok}[1]{#1}
\newcommand{\DocumentationTok}[1]{#1}
\newcommand{\AnnotationTok}[1]{#1}
\newcommand{\CommentVarTok}[1]{#1}
\newcommand{\OtherTok}[1]{#1}
\newcommand{\FunctionTok}[1]{#1}
\newcommand{\VariableTok}[1]{#1}
\newcommand{\ControlFlowTok}[1]{#1}
\newcommand{\OperatorTok}[1]{#1}
\newcommand{\BuiltInTok}[1]{#1}
\newcommand{\ExtensionTok}[1]{#1}
\newcommand{\PreprocessorTok}[1]{#1}
\newcommand{\AttributeTok}[1]{#1}
\newcommand{\RegionMarkerTok}[1]{#1}
\newcommand{\InformationTok}[1]{#1}
\newcommand{\WarningTok}[1]{#1}
\newcommand{\AlertTok}[1]{#1}
\newcommand{\ErrorTok}[1]{#1}
\newcommand{\NormalTok}[1]{#1}
% Redefine \includegraphics so that, unless explicit options are
% given, the image width will not exceed the width of the page.
% Images get their normal width if they fit onto the page, but
% are scaled down if they would overflow the margins.

\begin{document}

\maketitle
\tableofcontents

\hypertarget{premessa}{%
\chapter*{Premessa}\label{premessa}}
\addcontentsline{toc}{chapter}{Premessa}

Placeholder

\hypertarget{obiettivi}{%
\section*{Obiettivi}\label{obiettivi}}
\addcontentsline{toc}{section}{Obiettivi}

\hypertarget{organizzazione}{%
\section*{Organizzazione}\label{organizzazione}}
\addcontentsline{toc}{section}{Organizzazione}

\hypertarget{software-statistico}{%
\section*{Software statistico}\label{software-statistico}}
\addcontentsline{toc}{section}{Software statistico}

\hypertarget{the-authors}{%
\section*{The authors}\label{the-authors}}
\addcontentsline{toc}{section}{The authors}

\hypertarget{ringraziamenti}{%
\section*{Ringraziamenti}\label{ringraziamenti}}
\addcontentsline{toc}{section}{Ringraziamenti}

\hypertarget{scienza-e-pseudo-scienza}{%
\chapter{Scienza e pseudo-scienza}\label{scienza-e-pseudo-scienza}}

Placeholder

\hypertarget{scienza-dati}{%
\section{Scienza = dati}\label{scienza-dati}}

\hypertarget{dati-buoni-e-cattivi}{%
\section{Dati `buoni' e `cattivi'}\label{dati-buoni-e-cattivi}}

\hypertarget{dati-buoni-e-metodi-buoni}{%
\section{Dati `buoni' e metodi `buoni'}\label{dati-buoni-e-metodi-buoni}}

\hypertarget{il-principio-di-falsificazione}{%
\section{Il principio di falsificazione}\label{il-principio-di-falsificazione}}

\hypertarget{falsificare-un-risultato}{%
\section{Falsificare un risultato}\label{falsificare-un-risultato}}

\hypertarget{elementi-fondamentali-del-disegno-sperimentale}{%
\section{Elementi fondamentali del disegno sperimentale}\label{elementi-fondamentali-del-disegno-sperimentale}}

\hypertarget{controllo-degli-errori}{%
\subsection{Controllo degli errori}\label{controllo-degli-errori}}

\hypertarget{replicazione}{%
\subsection{Replicazione}\label{replicazione}}

\hypertarget{randomizzazione}{%
\subsection{Randomizzazione}\label{randomizzazione}}

\hypertarget{esperimenti-invalidi}{%
\subsection{Esperimenti invalidi}\label{esperimenti-invalidi}}

\hypertarget{cattivo-controllo-degli-errori}{%
\subsubsection{Cattivo controllo degli errori}\label{cattivo-controllo-degli-errori}}

\hypertarget{confounding-e-correlazione-spuria}{%
\subsubsection{`Confounding' e correlazione spuria}\label{confounding-e-correlazione-spuria}}

\hypertarget{pseudo-repliche-e-randomizzazione-poco-attenta}{%
\subsubsection{Pseudo-repliche e randomizzazione poco attenta}\label{pseudo-repliche-e-randomizzazione-poco-attenta}}

\hypertarget{chi-valuta-se-un-esperimento-uxe8-attendibile}{%
\section{Chi valuta se un esperimento è attendibile?}\label{chi-valuta-se-un-esperimento-uxe8-attendibile}}

\hypertarget{conclusioni}{%
\section{Conclusioni}\label{conclusioni}}

\hypertarget{altre-letture}{%
\section{Altre letture}\label{altre-letture}}

\hypertarget{progettare-un-esperimento}{%
\chapter{Progettare un esperimento}\label{progettare-un-esperimento}}

Placeholder

\hypertarget{gli-elementi-della-ricerca}{%
\section{Gli elementi della ricerca}\label{gli-elementi-della-ricerca}}

\hypertarget{ipotesi-scientifica-rightarrow-obiettivo-dellesperimento}{%
\section{\texorpdfstring{Ipotesi scientifica \(\rightarrow\) obiettivo dell'esperimento}{Ipotesi scientifica \textbackslash rightarrow obiettivo dell'esperimento}}\label{ipotesi-scientifica-rightarrow-obiettivo-dellesperimento}}

\hypertarget{identificazione-dei-fattori-sperimentali}{%
\section{Identificazione dei fattori sperimentali}\label{identificazione-dei-fattori-sperimentali}}

\hypertarget{esperimenti-multi-fattoriali}{%
\subsection{Esperimenti (multi-)fattoriali}\label{esperimenti-multi-fattoriali}}

\hypertarget{controllo-o-testimone}{%
\subsection{Controllo o testimone}\label{controllo-o-testimone}}

\hypertarget{le-unituxe0-sperimentali}{%
\section{Le unità sperimentali}\label{le-unituxe0-sperimentali}}

\hypertarget{allocazione-dei-trattamenti}{%
\section{Allocazione dei trattamenti}\label{allocazione-dei-trattamenti}}

\hypertarget{le-variabili-sperimentali}{%
\section{Le variabili sperimentali}\label{le-variabili-sperimentali}}

\hypertarget{variabili-nominali-categoriche}{%
\subsection{Variabili nominali (categoriche)}\label{variabili-nominali-categoriche}}

\hypertarget{variabili-ordinali}{%
\subsection{Variabili ordinali}\label{variabili-ordinali}}

\hypertarget{variabili-quantitative-discrete}{%
\subsection{Variabili quantitative discrete}\label{variabili-quantitative-discrete}}

\hypertarget{variabili-quantitative-continue}{%
\subsection{Variabili quantitative continue}\label{variabili-quantitative-continue}}

\hypertarget{rilievi-visivi-e-sensoriali}{%
\subsection{Rilievi visivi e sensoriali}\label{rilievi-visivi-e-sensoriali}}

\hypertarget{variabili-di-confondimento}{%
\subsection{Variabili di confondimento}\label{variabili-di-confondimento}}

\hypertarget{esperimenti-di-campo}{%
\section{Esperimenti di campo}\label{esperimenti-di-campo}}

\hypertarget{scegliere-il-campo}{%
\subsection{Scegliere il campo}\label{scegliere-il-campo}}

\hypertarget{le-unituxe0-sperimentali-in-campo}{%
\subsection{Le unità sperimentali in campo}\label{le-unituxe0-sperimentali-in-campo}}

\hypertarget{numero-di-repliche}{%
\subsection{Numero di repliche}\label{numero-di-repliche}}

\hypertarget{la-mappa-di-campo}{%
\subsection{La mappa di campo}\label{la-mappa-di-campo}}

\hypertarget{lay-out-sperimentale}{%
\subsection{Lay-out sperimentale}\label{lay-out-sperimentale}}

\hypertarget{disegni-completamente-randomizzati}{%
\subsubsection{Disegni completamente randomizzati}\label{disegni-completamente-randomizzati}}

\hypertarget{disegni-a-blocchi-randomizzati}{%
\subsubsection{Disegni a blocchi randomizzati}\label{disegni-a-blocchi-randomizzati}}

\hypertarget{disegni-a-quadrato-latino}{%
\subsubsection{Disegni a quadrato latino}\label{disegni-a-quadrato-latino}}

\hypertarget{disegni-a-split-plot}{%
\subsubsection{Disegni a split-plot}\label{disegni-a-split-plot}}

\hypertarget{disegni-a-strip-plot}{%
\subsubsection{Disegni a strip-plot}\label{disegni-a-strip-plot}}

\hypertarget{altre-letture-1}{%
\section{Altre letture}\label{altre-letture-1}}

\hypertarget{richiami-di-statistica-descrittiva}{%
\chapter{Richiami di statistica descrittiva}\label{richiami-di-statistica-descrittiva}}

Placeholder

\hypertarget{dati-quantitativi}{%
\section{Dati quantitativi}\label{dati-quantitativi}}

\hypertarget{indicatori-di-tendenza-centrale}{%
\subsection{Indicatori di tendenza centrale}\label{indicatori-di-tendenza-centrale}}

\hypertarget{indicatori-di-dispersione}{%
\subsection{Indicatori di dispersione}\label{indicatori-di-dispersione}}

\hypertarget{incertezza-delle-misure-derivate}{%
\subsection{Incertezza delle misure derivate}\label{incertezza-delle-misure-derivate}}

\hypertarget{relazioni-tra-variabili-quantitative-correlazione}{%
\subsection{Relazioni tra variabili quantitative: correlazione}\label{relazioni-tra-variabili-quantitative-correlazione}}

\hypertarget{dati-qualitativi}{%
\section{Dati qualitativi}\label{dati-qualitativi}}

\hypertarget{distribuzioni-di-frequenze-e-classamento}{%
\subsection{Distribuzioni di frequenze e classamento}\label{distribuzioni-di-frequenze-e-classamento}}

\hypertarget{statistiche-descrittive-per-le-distribuzioni-di-frequenze}{%
\subsection{Statistiche descrittive per le distribuzioni di frequenze}\label{statistiche-descrittive-per-le-distribuzioni-di-frequenze}}

\hypertarget{distribuzioni-di-frequenza-bivariate-le-tabelle-di-contingenze}{%
\subsection{Distribuzioni di frequenza bivariate: le tabelle di contingenze}\label{distribuzioni-di-frequenza-bivariate-le-tabelle-di-contingenze}}

\hypertarget{connessione}{%
\subsection{Connessione}\label{connessione}}

\hypertarget{statistiche-descrittive-con-r}{%
\section{Statistiche descrittive con R}\label{statistiche-descrittive-con-r}}

\hypertarget{descrizione-dei-sottogruppi}{%
\subsection{Descrizione dei sottogruppi}\label{descrizione-dei-sottogruppi}}

\hypertarget{distribuzioni-di-frequenze-e-classamento-1}{%
\subsection{Distribuzioni di frequenze e classamento}\label{distribuzioni-di-frequenze-e-classamento-1}}

\hypertarget{connessione-1}{%
\subsection{Connessione}\label{connessione-1}}

\hypertarget{altre-letture-2}{%
\section{Altre letture}\label{altre-letture-2}}

\hypertarget{modelli-statistici-ed-analisi-dei-dati}{%
\chapter{Modelli statistici ed analisi dei dati}\label{modelli-statistici-ed-analisi-dei-dati}}

Placeholder

\hypertarget{verituxe0-vera-e-modelli-deterministici}{%
\section{Verità `vera' e modelli deterministici}\label{verituxe0-vera-e-modelli-deterministici}}

\hypertarget{genesi-deterministica-delle-osservazioni-sperimentali}{%
\section{Genesi deterministica delle osservazioni sperimentali}\label{genesi-deterministica-delle-osservazioni-sperimentali}}

\hypertarget{errore-sperimentale-e-modelli-stocastici}{%
\section{Errore sperimentale e modelli stocastici}\label{errore-sperimentale-e-modelli-stocastici}}

\hypertarget{funzioni-di-probabilituxe0}{%
\subsection{Funzioni di probabilità}\label{funzioni-di-probabilituxe0}}

\hypertarget{funzioni-di-densituxe0}{%
\subsection{Funzioni di densità}\label{funzioni-di-densituxe0}}

\hypertarget{la-distribuzione-normale-curva-di-gauss}{%
\subsection{La distribuzione normale (curva di Gauss)}\label{la-distribuzione-normale-curva-di-gauss}}

\hypertarget{modelli-a-due-facce}{%
\section{Modelli `a due facce'}\label{modelli-a-due-facce}}

\hypertarget{e-allora}{%
\section{E allora?}\label{e-allora}}

\hypertarget{le-simulazioni-monte-carlo}{%
\section{Le simulazioni Monte Carlo}\label{le-simulazioni-monte-carlo}}

\hypertarget{analisi-dei-dati-e-model-fitting}{%
\section{Analisi dei dati e `model fitting'}\label{analisi-dei-dati-e-model-fitting}}

\hypertarget{modelli-stocastici-non-normali}{%
\section{Modelli stocastici non-normali}\label{modelli-stocastici-non-normali}}

\hypertarget{altre-letture-3}{%
\section{Altre letture}\label{altre-letture-3}}

\hypertarget{stime-ed-incertezza}{%
\chapter{Stime ed incertezza}\label{stime-ed-incertezza}}

Placeholder

\hypertarget{esempio-una-soluzione-erbicida}{%
\section{Esempio: una soluzione erbicida}\label{esempio-una-soluzione-erbicida}}

\hypertarget{analisi-dei-dati-stima-dei-parametri}{%
\subsection{Analisi dei dati: stima dei parametri}\label{analisi-dei-dati-stima-dei-parametri}}

\hypertarget{la-sampling-distribution}{%
\subsection{La `sampling distribution'}\label{la-sampling-distribution}}

\hypertarget{lerrore-standard}{%
\subsection{L'errore standard}\label{lerrore-standard}}

\hypertarget{stima-per-intervallo}{%
\section{Stima per intervallo}\label{stima-per-intervallo}}

\hypertarget{lintervallo-di-confidenza}{%
\section{L'intervallo di confidenza}\label{lintervallo-di-confidenza}}

\hypertarget{qual-uxe8-il-senso-dellintervallo-di-confidenza}{%
\section{Qual è il senso dell'intervallo di confidenza?}\label{qual-uxe8-il-senso-dellintervallo-di-confidenza}}

\hypertarget{come-presentare-i-risultati-degli-esperimenti}{%
\section{Come presentare i risultati degli esperimenti}\label{come-presentare-i-risultati-degli-esperimenti}}

\hypertarget{alcune-precisazioni}{%
\section{Alcune precisazioni}\label{alcune-precisazioni}}

\hypertarget{campioni-numerosi-e-non}{%
\subsection{Campioni numerosi e non}\label{campioni-numerosi-e-non}}

\hypertarget{popolazioni-gaussiane-e-non}{%
\subsection{Popolazioni gaussiane e non}\label{popolazioni-gaussiane-e-non}}

\hypertarget{analisi-statistica-dei-dati-riassunto-del-percorso-logico}{%
\section{Analisi statistica dei dati: riassunto del percorso logico}\label{analisi-statistica-dei-dati-riassunto-del-percorso-logico}}

\hypertarget{da-ricordare}{%
\section{Da ricordare}\label{da-ricordare}}

\hypertarget{per-approfondire-un-po}{%
\section{Per approfondire un po'\ldots{}}\label{per-approfondire-un-po}}

\hypertarget{coverage-degli-intervalli-di-confidenza}{%
\section{\texorpdfstring{\emph{Coverage} degli intervalli di confidenza}{Coverage degli intervalli di confidenza}}\label{coverage-degli-intervalli-di-confidenza}}

\hypertarget{intervalli-di-confidenza-per-fenomeni-non-normali}{%
\subsection{Intervalli di confidenza per fenomeni non-normali}\label{intervalli-di-confidenza-per-fenomeni-non-normali}}

\hypertarget{altre-letture-4}{%
\section{Altre letture}\label{altre-letture-4}}

\hypertarget{decisioni-ed-incertezza}{%
\chapter{Decisioni ed incertezza}\label{decisioni-ed-incertezza}}

Placeholder

\hypertarget{confronto-tra-due-medie-il-test-t-di-student}{%
\section{Confronto tra due medie: il test t di Student}\label{confronto-tra-due-medie-il-test-t-di-student}}

\hypertarget{lipotesi-nulla-e-alternativa}{%
\subsection{L'ipotesi nulla e alternativa}\label{lipotesi-nulla-e-alternativa}}

\hypertarget{la-statistica-t}{%
\subsection{La statistica T}\label{la-statistica-t}}

\hypertarget{simulazione-monte-carlo}{%
\subsection{Simulazione Monte Carlo}\label{simulazione-monte-carlo}}

\hypertarget{soluzione-formale}{%
\subsection{Soluzione formale}\label{soluzione-formale}}

\hypertarget{interpretazione-del-p-level}{%
\subsection{Interpretazione del P-level}\label{interpretazione-del-p-level}}

\hypertarget{tipologie-alternative-di-test-t-di-student}{%
\subsection{Tipologie alternative di test t di Student}\label{tipologie-alternative-di-test-t-di-student}}

\hypertarget{confronto-tra-due-proporzioni-il-test-chi2}{%
\section{\texorpdfstring{Confronto tra due proporzioni: il test \(\chi^2\)}{Confronto tra due proporzioni: il test \textbackslash chi\^{}2}}\label{confronto-tra-due-proporzioni-il-test-chi2}}

\hypertarget{simulazione-monte-carlo-1}{%
\subsection{Simulazione Monte Carlo}\label{simulazione-monte-carlo-1}}

\hypertarget{soluzione-formale-1}{%
\subsection{Soluzione formale}\label{soluzione-formale-1}}

\hypertarget{conclusioni-e-riepilogo}{%
\section{Conclusioni e riepilogo}\label{conclusioni-e-riepilogo}}

\hypertarget{altre-letture-5}{%
\section{Altre letture}\label{altre-letture-5}}

\hypertarget{modelli-anova-ad-una-via}{%
\chapter{Modelli ANOVA ad una via}\label{modelli-anova-ad-una-via}}

Placeholder

\hypertarget{caso-studio-confronto-tra-erbicidi-in-vaso}{%
\section{Caso-studio: confronto tra erbicidi in vaso}\label{caso-studio-confronto-tra-erbicidi-in-vaso}}

\hypertarget{descrizione-del-dataset}{%
\section{Descrizione del dataset}\label{descrizione-del-dataset}}

\hypertarget{definizione-di-un-modello-lineare}{%
\section{Definizione di un modello lineare}\label{definizione-di-un-modello-lineare}}

\hypertarget{parametrizzazione-del-modello}{%
\section{Parametrizzazione del modello}\label{parametrizzazione-del-modello}}

\hypertarget{assunzioni-di-base}{%
\section{Assunzioni di base}\label{assunzioni-di-base}}

\hypertarget{fitting-del-modello-metodo-manuale}{%
\section{Fitting del modello: metodo manuale}\label{fitting-del-modello-metodo-manuale}}

\hypertarget{stima-dei-parametri}{%
\subsection{Stima dei parametri}\label{stima-dei-parametri}}

\hypertarget{calcolo-dei-residui}{%
\subsection{Calcolo dei residui}\label{calcolo-dei-residui}}

\hypertarget{stima-di-sigma}{%
\subsection{\texorpdfstring{Stima di \(\sigma\)}{Stima di \textbackslash sigma}}\label{stima-di-sigma}}

\hypertarget{scomposizione-della-varianza}{%
\section{Scomposizione della varianza}\label{scomposizione-della-varianza}}

\hypertarget{test-dipotesi}{%
\section{Test d'ipotesi}\label{test-dipotesi}}

\hypertarget{inferenza-statistica}{%
\section{Inferenza statistica}\label{inferenza-statistica}}

\hypertarget{fitting-del-modello-con-r}{%
\section{Fitting del modello con R}\label{fitting-del-modello-con-r}}

\hypertarget{medie-marginali-attese}{%
\section{Medie marginali attese}\label{medie-marginali-attese}}

\hypertarget{per-concludere}{%
\section{Per concludere \ldots{}}\label{per-concludere}}

\hypertarget{altre-letture-6}{%
\section{Altre letture}\label{altre-letture-6}}

\hypertarget{la-verifica-delle-assunzioni-di-base}{%
\chapter{La verifica delle assunzioni di base}\label{la-verifica-delle-assunzioni-di-base}}

Placeholder

\hypertarget{violazioni-delle-assunzioni-di-base}{%
\section{Violazioni delle assunzioni di base}\label{violazioni-delle-assunzioni-di-base}}

\hypertarget{procedure-diagnostiche}{%
\section{Procedure diagnostiche}\label{procedure-diagnostiche}}

\hypertarget{analisi-grafica-dei-residui}{%
\section{Analisi grafica dei residui}\label{analisi-grafica-dei-residui}}

\hypertarget{grafico-dei-residui-contro-i-valori-attesi}{%
\subsection{Grafico dei residui contro i valori attesi}\label{grafico-dei-residui-contro-i-valori-attesi}}

\hypertarget{qq-plot}{%
\subsection{QQ-plot}\label{qq-plot}}

\hypertarget{test-dipotesi-1}{%
\section{Test d'ipotesi}\label{test-dipotesi-1}}

\hypertarget{risultati-contraddittori}{%
\section{Risultati contraddittori}\label{risultati-contraddittori}}

\hypertarget{terapia}{%
\section{`Terapia'}\label{terapia}}

\hypertarget{correzionerimozione-degli-outliers}{%
\subsection{Correzione/Rimozione degli outliers}\label{correzionerimozione-degli-outliers}}

\hypertarget{correzione-del-modello}{%
\subsection{Correzione del modello}\label{correzione-del-modello}}

\hypertarget{trasformazione-della-variabile-indipendente}{%
\subsection{Trasformazione della variabile indipendente}\label{trasformazione-della-variabile-indipendente}}

\hypertarget{impiego-di-metodiche-statistiche-avanzate}{%
\subsection{Impiego di metodiche statistiche avanzate}\label{impiego-di-metodiche-statistiche-avanzate}}

\hypertarget{trasformazioni-stabilizzanti}{%
\subsection{Trasformazioni stabilizzanti}\label{trasformazioni-stabilizzanti}}

\hypertarget{esempio-1}{%
\section{Esempio 1}\label{esempio-1}}

\hypertarget{esempio-2}{%
\section{Esempio 2}\label{esempio-2}}

\hypertarget{altre-letture-7}{%
\section{Altre letture}\label{altre-letture-7}}

\hypertarget{contrasti-e-confronti-multipli}{%
\chapter{Contrasti e confronti multipli}\label{contrasti-e-confronti-multipli}}

Placeholder

\hypertarget{esempio}{%
\section{Esempio}\label{esempio}}

\hypertarget{i-contrasti}{%
\section{I contrasti}\label{i-contrasti}}

\hypertarget{i-contrasti-con-r}{%
\section{I contrasti con R}\label{i-contrasti-con-r}}

\hypertarget{i-confronti-multipli-a-coppie-pairwise-comparisons}{%
\section{I confronti multipli a coppie (pairwise comparisons)}\label{i-confronti-multipli-a-coppie-pairwise-comparisons}}

\hypertarget{display-a-lettere}{%
\section{Display a lettere}\label{display-a-lettere}}

\hypertarget{tassi-di-errore-per-confronto-e-per-esperimento}{%
\section{Tassi di errore per confronto e per esperimento}\label{tassi-di-errore-per-confronto-e-per-esperimento}}

\hypertarget{aggiustamento-per-la-molteplicituxe0}{%
\section{Aggiustamento per la molteplicità}\label{aggiustamento-per-la-molteplicituxe0}}

\hypertarget{e-le-classiche-procedure-di-confronto-multiplo}{%
\section{E le classiche procedure di confronto multiplo?}\label{e-le-classiche-procedure-di-confronto-multiplo}}

\hypertarget{consigli-pratici}{%
\section{Consigli pratici}\label{consigli-pratici}}

\hypertarget{altre-letture-8}{%
\section{Altre letture}\label{altre-letture-8}}

\hypertarget{modelli-anova-con-fattori-di-blocco}{%
\chapter{Modelli ANOVA con fattori di blocco}\label{modelli-anova-con-fattori-di-blocco}}

Placeholder

\hypertarget{caso-studio-confronto-tra-erbicidi-in-campo}{%
\section{Caso-studio: confronto tra erbicidi in campo}\label{caso-studio-confronto-tra-erbicidi-in-campo}}

\hypertarget{definizione-di-un-modello-lineare-1}{%
\section{Definizione di un modello lineare}\label{definizione-di-un-modello-lineare-1}}

\hypertarget{stima-dei-parametri-1}{%
\section{Stima dei parametri}\label{stima-dei-parametri-1}}

\hypertarget{coefficienti-del-modello}{%
\subsection{Coefficienti del modello}\label{coefficienti-del-modello}}

\hypertarget{stima-di-sigma-1}{%
\subsection{\texorpdfstring{Stima di \(\sigma\)}{Stima di \textbackslash sigma}}\label{stima-di-sigma-1}}

\hypertarget{scomposizione-della-varianza-1}{%
\section{Scomposizione della varianza}\label{scomposizione-della-varianza-1}}

\hypertarget{adattamento-del-modello-con-r}{%
\section{Adattamento del modello con R}\label{adattamento-del-modello-con-r}}

\hypertarget{disegni-a-quadrato-latino-1}{%
\section{Disegni a quadrato latino}\label{disegni-a-quadrato-latino-1}}

\hypertarget{caso-studio-confronto-tra-metodi-costruttivi}{%
\section{Caso studio: confronto tra metodi costruttivi}\label{caso-studio-confronto-tra-metodi-costruttivi}}

\hypertarget{definizione-di-un-modello-lineare-2}{%
\section{Definizione di un modello lineare}\label{definizione-di-un-modello-lineare-2}}

\hypertarget{la-regressione-lineare-semplice}{%
\chapter{La regressione lineare semplice}\label{la-regressione-lineare-semplice}}

Placeholder

\hypertarget{caso-studio-effetto-della-concimazione-azotata-al-frumento}{%
\section{Caso studio: effetto della concimazione azotata al frumento}\label{caso-studio-effetto-della-concimazione-azotata-al-frumento}}

\hypertarget{analisi-preliminari}{%
\section{Analisi preliminari}\label{analisi-preliminari}}

\hypertarget{definizione-del-modello-lineare}{%
\section{Definizione del modello lineare}\label{definizione-del-modello-lineare}}

\hypertarget{stima-dei-parametri-2}{%
\section{Stima dei parametri}\label{stima-dei-parametri-2}}

\hypertarget{valutazione-della-bontuxe0-del-modello}{%
\section{Valutazione della bontà del modello}\label{valutazione-della-bontuxe0-del-modello}}

\hypertarget{valutazione-grafica}{%
\subsection{Valutazione grafica}\label{valutazione-grafica}}

\hypertarget{errori-standard-dei-parametri}{%
\subsection{Errori standard dei parametri}\label{errori-standard-dei-parametri}}

\hypertarget{test-f-per-la-mancanza-dadattamento}{%
\subsection{Test F per la mancanza d'adattamento}\label{test-f-per-la-mancanza-dadattamento}}

\hypertarget{test-f-per-la-bontuxe0-di-adattamento-e-coefficiente-di-determinazione}{%
\subsection{Test F per la bontà di adattamento e coefficiente di determinazione}\label{test-f-per-la-bontuxe0-di-adattamento-e-coefficiente-di-determinazione}}

\hypertarget{previsioni}{%
\section{Previsioni}\label{previsioni}}

\hypertarget{altre-letture-9}{%
\section{Altre letture}\label{altre-letture-9}}

\hypertarget{modelli-anova-a-due-vie-con-interazione}{%
\chapter{Modelli ANOVA a due vie con interazione}\label{modelli-anova-a-due-vie-con-interazione}}

Placeholder

\hypertarget{il-concetto-di-interazione}{%
\section{Il concetto di 'interazione'}\label{il-concetto-di-interazione}}

\hypertarget{effetti-incrociati-interazione-tra-lavorazioni-e-diserbo-chimico}{%
\section{Effetti incrociati: interazione tra lavorazioni e diserbo chimico}\label{effetti-incrociati-interazione-tra-lavorazioni-e-diserbo-chimico}}

\hypertarget{definizione-del-modello-lineare-1}{%
\section{Definizione del modello lineare}\label{definizione-del-modello-lineare-1}}

\hypertarget{calcoli-manuali}{%
\section{Calcoli manuali}\label{calcoli-manuali}}

\hypertarget{scomposizione-della-varianza-2}{%
\subsection{Scomposizione della varianza}\label{scomposizione-della-varianza-2}}

\hypertarget{calcoli-con-r}{%
\section{Calcoli con R}\label{calcoli-con-r}}

\hypertarget{model-fitting}{%
\subsection{Model fitting}\label{model-fitting}}

\hypertarget{verifica-delle-assunzioni-di-base}{%
\subsection{Verifica delle assunzioni di base}\label{verifica-delle-assunzioni-di-base}}

\hypertarget{scomposizione-della-varianza-3}{%
\subsection{Scomposizione della varianza}\label{scomposizione-della-varianza-3}}

\hypertarget{medie-marginali-attese-e-confronti-multipli-con-r}{%
\subsection{Medie marginali attese e confronti multipli con R}\label{medie-marginali-attese-e-confronti-multipli-con-r}}

\hypertarget{effetti-innestati-valutazione-di-ibridi-di-mais}{%
\section{Effetti innestati: valutazione di ibridi di mais}\label{effetti-innestati-valutazione-di-ibridi-di-mais}}

\hypertarget{definizione-del-modello-lineare-2}{%
\section{Definizione del modello lineare}\label{definizione-del-modello-lineare-2}}

\hypertarget{fitting-del-modello-con-r-1}{%
\section{Fitting del modello con R}\label{fitting-del-modello-con-r-1}}

\hypertarget{split-plots-strip-plots-e-altri-disegni-sperimentali-in-campo}{%
\chapter{Split-plots, strip-plots e altri disegni sperimentali in campo}\label{split-plots-strip-plots-e-altri-disegni-sperimentali-in-campo}}

Placeholder

\hypertarget{raggruppamenti-tra-parcelle}{%
\section{Raggruppamenti tra parcelle}\label{raggruppamenti-tra-parcelle}}

\hypertarget{esperimenti-a-split-plot}{%
\section{Esperimenti a split-plot}\label{esperimenti-a-split-plot}}

\hypertarget{definizione-del-modello-lineare-3}{%
\subsection{Definizione del modello lineare}\label{definizione-del-modello-lineare-3}}

\hypertarget{model-fitting-con-r}{%
\subsection{Model fitting con R}\label{model-fitting-con-r}}

\hypertarget{esperimenti-a-strip-plot}{%
\section{Esperimenti a strip-plot}\label{esperimenti-a-strip-plot}}

\hypertarget{definizione-del-modello-lineare-4}{%
\subsection{Definizione del modello lineare}\label{definizione-del-modello-lineare-4}}

\hypertarget{model-fitting-con-r-1}{%
\subsection{Model fitting con R}\label{model-fitting-con-r-1}}

\hypertarget{altre-letture-10}{%
\section{Altre letture}\label{altre-letture-10}}

\hypertarget{la-regressione-non-lineare}{%
\chapter{La regressione non-lineare}\label{la-regressione-non-lineare}}

Placeholder

\hypertarget{caso-studio-degradazione-di-un-erbicida-nel-terreno}{%
\section{Caso studio: degradazione di un erbicida nel terreno}\label{caso-studio-degradazione-di-un-erbicida-nel-terreno}}

\hypertarget{scelta-della-funzione}{%
\section{Scelta della funzione}\label{scelta-della-funzione}}

\hypertarget{stima-dei-parametri-3}{%
\section{Stima dei parametri}\label{stima-dei-parametri-3}}

\hypertarget{linearizzazione-della-funzione}{%
\subsection{Linearizzazione della funzione}\label{linearizzazione-della-funzione}}

\hypertarget{approssimazione-della-vera-funzione-tramite-una-polinomiale-in-x}{%
\subsection{Approssimazione della vera funzione tramite una polinomiale in X}\label{approssimazione-della-vera-funzione-tramite-una-polinomiale-in-x}}

\hypertarget{minimi-quadrati-non-lineari}{%
\subsection{Minimi quadrati non-lineari}\label{minimi-quadrati-non-lineari}}

\hypertarget{la-regressione-non-lineare-con-r}{%
\section{La regressione non-lineare con R}\label{la-regressione-non-lineare-con-r}}

\hypertarget{verifica-della-bontuxe0-del-modello}{%
\section{Verifica della bontà del modello}\label{verifica-della-bontuxe0-del-modello}}

\hypertarget{analisi-grafica-dei-residui-1}{%
\subsection{Analisi grafica dei residui}\label{analisi-grafica-dei-residui-1}}

\hypertarget{test-f-per-la-mancanza-di-adattamento-approssimato}{%
\subsection{Test F per la mancanza di adattamento (approssimato)}\label{test-f-per-la-mancanza-di-adattamento-approssimato}}

\hypertarget{errori-standard-dei-parametri-1}{%
\subsection{Errori standard dei parametri}\label{errori-standard-dei-parametri-1}}

\hypertarget{coefficienti-di-determinazione}{%
\subsection{Coefficienti di determinazione}\label{coefficienti-di-determinazione}}

\hypertarget{funzioni-lineari-e-nonlineari-dei-parametri}{%
\section{Funzioni lineari e nonlineari dei parametri}\label{funzioni-lineari-e-nonlineari-dei-parametri}}

\hypertarget{previsioni-1}{%
\section{Previsioni}\label{previsioni-1}}

\hypertarget{gestione-delle-situazioni-patologiche}{%
\section{Gestione delle situazioni `patologiche'}\label{gestione-delle-situazioni-patologiche}}

\hypertarget{trasformazione-del-modello}{%
\subsection{Trasformazione del modello}\label{trasformazione-del-modello}}

\hypertarget{trasformazione-dei-dati}{%
\subsection{Trasformazione dei dati}\label{trasformazione-dei-dati}}

\hypertarget{per-approfondire-un-po-1}{%
\section{Per approfondire un po'\ldots{}}\label{per-approfondire-un-po-1}}

\hypertarget{riparametrizzazione-delle-funzioni-non-lineari}{%
\subsection{Riparametrizzazione delle funzioni non-lineari}\label{riparametrizzazione-delle-funzioni-non-lineari}}

\hypertarget{altre-letture-11}{%
\subsection{Altre letture}\label{altre-letture-11}}

\hypertarget{esercizi}{%
\chapter{Esercizi}\label{esercizi}}

Placeholder

\hypertarget{capitoli-1-e-2}{%
\section{Capitoli 1 e 2}\label{capitoli-1-e-2}}

\hypertarget{esercizio-1}{%
\subsection{Esercizio 1}\label{esercizio-1}}

\hypertarget{capitolo-3}{%
\section{Capitolo 3}\label{capitolo-3}}

\hypertarget{esercizio-1-1}{%
\subsection{Esercizio 1}\label{esercizio-1-1}}

\hypertarget{esercizio-2}{%
\subsection{Esercizio 2}\label{esercizio-2}}

\hypertarget{esercizio-3}{%
\subsection{Esercizio 3}\label{esercizio-3}}

\hypertarget{capitolo-4}{%
\section{Capitolo 4}\label{capitolo-4}}

\hypertarget{esercizio-1-2}{%
\subsection{Esercizio 1}\label{esercizio-1-2}}

\hypertarget{esercizio-2-1}{%
\subsection{Esercizio 2}\label{esercizio-2-1}}

\hypertarget{esercizio-3-1}{%
\subsection{Esercizio 3}\label{esercizio-3-1}}

\hypertarget{esercizio-4}{%
\subsection{Esercizio 4}\label{esercizio-4}}

\hypertarget{esercizio-5}{%
\subsection{Esercizio 5}\label{esercizio-5}}

\hypertarget{esercizio-6}{%
\subsection{Esercizio 6}\label{esercizio-6}}

\hypertarget{esercizio-7}{%
\subsection{Esercizio 7}\label{esercizio-7}}

\hypertarget{esercizio-8}{%
\subsection{Esercizio 8}\label{esercizio-8}}

\hypertarget{capitolo-5}{%
\section{Capitolo 5}\label{capitolo-5}}

\hypertarget{esercizio-1-3}{%
\subsection{Esercizio 1}\label{esercizio-1-3}}

\hypertarget{esercizio-2-2}{%
\subsection{Esercizio 2}\label{esercizio-2-2}}

\hypertarget{esercizio-3-2}{%
\subsection{Esercizio 3}\label{esercizio-3-2}}

\hypertarget{esercizio-4-1}{%
\subsection{Esercizio 4}\label{esercizio-4-1}}

\hypertarget{esercizio-5-1}{%
\subsection{Esercizio 5}\label{esercizio-5-1}}

\hypertarget{capitolo-6}{%
\section{Capitolo 6}\label{capitolo-6}}

\hypertarget{esercizio-1-4}{%
\subsection{Esercizio 1}\label{esercizio-1-4}}

\hypertarget{esercizio-2-3}{%
\subsection{Esercizio 2}\label{esercizio-2-3}}

\hypertarget{esercizio-3-3}{%
\subsection{Esercizio 3}\label{esercizio-3-3}}

\hypertarget{esercizio-4-2}{%
\subsection{Esercizio 4}\label{esercizio-4-2}}

\hypertarget{esercizio-5-2}{%
\subsection{Esercizio 5}\label{esercizio-5-2}}

\hypertarget{esercizio-6-1}{%
\subsection{Esercizio 6}\label{esercizio-6-1}}

\hypertarget{esercizio-7-1}{%
\subsection{Esercizio 7}\label{esercizio-7-1}}

\hypertarget{esercizio-8-1}{%
\subsection{Esercizio 8}\label{esercizio-8-1}}

\hypertarget{esercizio-9}{%
\subsection{Esercizio 9}\label{esercizio-9}}

\hypertarget{esercizio-10}{%
\subsection{Esercizio 10}\label{esercizio-10}}

\hypertarget{capitoli-da-7-a-9}{%
\section{Capitoli da 7 a 9}\label{capitoli-da-7-a-9}}

\hypertarget{esercizio-1-5}{%
\subsection{Esercizio 1}\label{esercizio-1-5}}

\hypertarget{esercizio-2-4}{%
\subsection{Esercizio 2}\label{esercizio-2-4}}

\hypertarget{esercizio-3-4}{%
\subsection{Esercizio 3}\label{esercizio-3-4}}

\hypertarget{esercizio-4-3}{%
\subsection{Esercizio 4}\label{esercizio-4-3}}

\hypertarget{capitolo-10}{%
\section{Capitolo 10}\label{capitolo-10}}

\hypertarget{esercizio-1-6}{%
\subsection{Esercizio 1}\label{esercizio-1-6}}

\hypertarget{esercizio-2-5}{%
\subsection{Esercizio 2}\label{esercizio-2-5}}

\hypertarget{esercizio-3-5}{%
\subsection{Esercizio 3}\label{esercizio-3-5}}

\hypertarget{capitolo-11}{%
\section{Capitolo 11}\label{capitolo-11}}

\hypertarget{esercizio-1-7}{%
\subsection{Esercizio 1}\label{esercizio-1-7}}

\hypertarget{esercizio-2-6}{%
\subsection{Esercizio 2}\label{esercizio-2-6}}

\hypertarget{capitoli-12-e-13}{%
\section{Capitoli 12 e 13}\label{capitoli-12-e-13}}

\hypertarget{esercizio-1-8}{%
\subsection{Esercizio 1}\label{esercizio-1-8}}

\hypertarget{esercizio-2-7}{%
\subsection{Esercizio 2}\label{esercizio-2-7}}

\hypertarget{esercizio-3-6}{%
\subsection{Esercizio 3}\label{esercizio-3-6}}

\hypertarget{esercizio-4-4}{%
\subsection{Esercizio 4}\label{esercizio-4-4}}

\hypertarget{esercizio-5-3}{%
\subsection{Esercizio 5}\label{esercizio-5-3}}

\hypertarget{esercizio-6-2}{%
\subsection{Esercizio 6}\label{esercizio-6-2}}

\hypertarget{capitolo-14}{%
\section{Capitolo 14}\label{capitolo-14}}

\hypertarget{esercizio-1-9}{%
\subsection{Esercizio 1}\label{esercizio-1-9}}

\hypertarget{esercizio-2-8}{%
\subsection{Esercizio 2}\label{esercizio-2-8}}

\hypertarget{esercizio-3-7}{%
\subsection{Esercizio 3}\label{esercizio-3-7}}

\hypertarget{esercizio-4-5}{%
\subsection{Esercizio 4}\label{esercizio-4-5}}

\hypertarget{esercizio-5-4}{%
\subsection{Esercizio 5}\label{esercizio-5-4}}

\hypertarget{esercizio-6-3}{%
\subsection{Esercizio 6}\label{esercizio-6-3}}

\hypertarget{esercizio-7-2}{%
\subsection{Esercizio 7}\label{esercizio-7-2}}

\hypertarget{appendice-1-breve-introduzione-ad-r}{%
\chapter{Appendice 1: breve introduzione ad R}\label{appendice-1-breve-introduzione-ad-r}}

Placeholder

\hypertarget{cosa-uxe8-r}{%
\section*{Cosa è R?}\label{cosa-uxe8-r}}
\addcontentsline{toc}{section}{Cosa è R?}

\hypertarget{oggetti-e-assegnazioni}{%
\section*{Oggetti e assegnazioni}\label{oggetti-e-assegnazioni}}
\addcontentsline{toc}{section}{Oggetti e assegnazioni}

\hypertarget{costanti-e-vettori}{%
\subsection*{Costanti e vettori}\label{costanti-e-vettori}}
\addcontentsline{toc}{subsection}{Costanti e vettori}

\hypertarget{matrici}{%
\subsection*{Matrici}\label{matrici}}
\addcontentsline{toc}{subsection}{Matrici}

\hypertarget{dataframe}{%
\subsection*{Dataframe}\label{dataframe}}
\addcontentsline{toc}{subsection}{Dataframe}

\hypertarget{quale-oggetto-sto-utilizzando}{%
\subsection*{Quale oggetto sto utilizzando?}\label{quale-oggetto-sto-utilizzando}}
\addcontentsline{toc}{subsection}{Quale oggetto sto utilizzando?}

\hypertarget{operazioni-ed-operatori}{%
\section*{Operazioni ed operatori}\label{operazioni-ed-operatori}}
\addcontentsline{toc}{section}{Operazioni ed operatori}

\hypertarget{funzioni-ed-argomenti}{%
\section*{Funzioni ed argomenti}\label{funzioni-ed-argomenti}}
\addcontentsline{toc}{section}{Funzioni ed argomenti}

\hypertarget{consigli-per-limmissione-di-dati-sperimentali}{%
\section*{Consigli per l'immissione di dati sperimentali}\label{consigli-per-limmissione-di-dati-sperimentali}}
\addcontentsline{toc}{section}{Consigli per l'immissione di dati sperimentali}

\hypertarget{immissione-di-numeri-progressivi}{%
\subsection*{Immissione di numeri progressivi}\label{immissione-di-numeri-progressivi}}
\addcontentsline{toc}{subsection}{Immissione di numeri progressivi}

\hypertarget{immissione-dei-codici-delle-tesi-e-dei-blocchi}{%
\subsection*{Immissione dei codici delle tesi e dei blocchi}\label{immissione-dei-codici-delle-tesi-e-dei-blocchi}}
\addcontentsline{toc}{subsection}{Immissione dei codici delle tesi e dei blocchi}

\hypertarget{immissione-dei-valori-e-creazione-del-datframe}{%
\subsection*{Immissione dei valori e creazione del datframe}\label{immissione-dei-valori-e-creazione-del-datframe}}
\addcontentsline{toc}{subsection}{Immissione dei valori e creazione del datframe}

\hypertarget{leggere-e-salvare-dati-esterni}{%
\subsection*{Leggere e salvare dati esterni}\label{leggere-e-salvare-dati-esterni}}
\addcontentsline{toc}{subsection}{Leggere e salvare dati esterni}

\hypertarget{alcune-operazioni-comuni-sul-dataset}{%
\section*{Alcune operazioni comuni sul dataset}\label{alcune-operazioni-comuni-sul-dataset}}
\addcontentsline{toc}{section}{Alcune operazioni comuni sul dataset}

\hypertarget{selezionare-un-subset-di-dati}{%
\subsection*{Selezionare un subset di dati}\label{selezionare-un-subset-di-dati}}
\addcontentsline{toc}{subsection}{Selezionare un subset di dati}

\hypertarget{ordinare-un-vettore-o-un-dataframe}{%
\subsection*{Ordinare un vettore o un dataframe}\label{ordinare-un-vettore-o-un-dataframe}}
\addcontentsline{toc}{subsection}{Ordinare un vettore o un dataframe}

\hypertarget{workspace}{%
\section*{Workspace}\label{workspace}}
\addcontentsline{toc}{section}{Workspace}

\hypertarget{script-o-programmi}{%
\section*{Script o programmi}\label{script-o-programmi}}
\addcontentsline{toc}{section}{Script o programmi}

\hypertarget{interrogazione-di-oggetti}{%
\section*{Interrogazione di oggetti}\label{interrogazione-di-oggetti}}
\addcontentsline{toc}{section}{Interrogazione di oggetti}

\hypertarget{altre-funzioni-matriciali}{%
\section*{Altre funzioni matriciali}\label{altre-funzioni-matriciali}}
\addcontentsline{toc}{section}{Altre funzioni matriciali}

\hypertarget{cenni-sulle-funzionalituxe0-grafiche-in-r}{%
\section*{Cenni sulle funzionalità grafiche in R}\label{cenni-sulle-funzionalituxe0-grafiche-in-r}}
\addcontentsline{toc}{section}{Cenni sulle funzionalità grafiche in R}

\hypertarget{altre-letture-12}{%
\section*{Altre letture}\label{altre-letture-12}}
\addcontentsline{toc}{section}{Altre letture}


\end{document}
